% $Id: misc.tex,v 1.47 2013/01/09 19:40:59 billl Exp $

% \usepackage{ifthen}
% \usepackage{mycite}
% \usepackage{amsmath}

\ifx\graphicspath\undefined
  \ifx\pdfoutput\undefined
    \usepackage[dvips]{graphicx}
    \DeclareGraphicsExtensions{.ps,.id,.eps}
  \else
    \usepackage[pdftex]{graphicx}
    \DeclareGraphicsExtensions{.pdf}
    \usepackage{type1cm}
  \fi
  \graphicspath{{id/}{figs/}{ps/}{pdf/}{gif/}{img/}{gif2/}}
\else
  \message{**** include misc before graphicx? ****} %%% or use \message
\fi

% convenient functions
\def\bpath{/usr/site/config/papers} %%%% site of our .bib files

% scientific notation (sn) 
\newcommand{\sn}[2]{$\mathrm #1 \cdot 10^{#2} $}
% \Exp1.1(4)
\def\Exp#1(#2){\ensuremath{#1 \cdot 10^{#2}}}
% \e1.1e4
\def\e#1e#2{{#1} $\!\cdot10^{#2}$}
% \p1.1e2,4
\def\p#1e#2,#3{\ensuremath{\mathrm #1 \cdot #2^{#3}}}

\newcommand{\code}[1]{\\ {\tt #1} \\ \noindent}
\newcommand{\nibf}[1]{\noindent{\bf #1}}
\newcommand{\libf}[1]{\vspace*{0.1in} \noindent {\Large \bf #1}}
\newcommand{\niem}[1]{\noindent{\em #1}}
\def\ra{$\rightarrow$}
\def\etal{{\em et al.}}
\def\cf{{\em cf.}}
\def\ie{{\em i.e.,}}
\def\Ie{{\em I.e.,}}
\def\eg{{\em e.g.,}}
\def\etc{{\em etc}}
\def\Eg{{\em E.g.,}}
\newcounter{c2} 
\newcounter{cnt}
\newcounter{fig}
\setcounter{fig}{0}
\setcounter{c2}{0}
\def\inc0{\setcounter{c2}{0}}
\def\incn{\noindent \stepcounter{c2}{1}\arabic{c2}. }

\def\sstretch{1.0}
\newcommand{\newstretch}[1]{\setstretch{#1} \small\normalsize}
\newcommand{\restretch}[0]{\setstretch{\sstretch} \small\normalsize}
\newcommand{\float}[1]{\parbox[htb][#1in][0in]{6in}{~~~}

}

\def\bpm{\begin{pmatrix}}    \def\epm{\end{pmatrix}}
\def\beq{\begin{equation}}   \def\eeq{\end{equation}}
\def\be{\begin{equation*}}   \def\ee{\end{equation*}}
\def\bip{\begin{inparaenum}[\bfseries{}1.]} \def\eip{\end{inparaenum}} 
\def\bipp{\begin{inparaenum}[\itshape{}a.]} %%%% for nested listing 

\newcommand{\bfig}[3]{
\begin{figure}[htb]
\begin{center}
\epsfig{figure=#1,height=#2in,width=#3in}
\end{center}
}
\newcommand{\efig}[0]{\end{figure}}

\newcommand{\bfigg}[2]{
\begin{figure}[htb]
\begin{center}
\epsfig{figure=#1,height=#2in}
\end{center}
}

% formatting
\ifx\note\undefined
  \newcommand{\note}[1]{} % Create a way of putting in in-text notes
\else
  \renewcommand{\note}[1]{} % need to redefine \note{} anyway in case someone redefined it with diff args
\fi
\newcommand{\BAD}[1]{#1} % if using the wrong word identify the fact with BAD
\newcommand{\NOTE}[1]{} % more visible note
\newcommand{\notnote}[1]{{#1}} % something in process of going back and forth
\newcommand{\Note}[1]{\hspace*{2in}{\em \bf $\langle\langle$ Note: #1 $\rangle\rangle$ }} % show a note
\newcommand{\mydef}[2]{\global\def#1{#2}{#2}} % see also \suba in papers/grant/m1u01/ctl.tex
\newcommand{\new}[1]{#1}

% begin and end numbered list
\newcommand{\blist}[0]{\begin{list}{\arabic{cnt}.}{\usecounter{cnt}}}
\newcommand{\elist}[0]{\end{list}}
\def\bbar{\cbstart}
\def\ebar{\cbend}

% figures
\newcommand{\pref}[1]{(see p. \pageref{#1})}
\newcommand{\cref}[1]{Chap.~\ref{#1}}
\newcommand{\eref}[1]{Eqn.~\ref{#1}}
\newcommand{\fref}[1]{Fig.~\ref{#1}}
\newcommand{\bfref}[1]{\noindent {\bf Figure~\ref{#1}}}
\newcommand{\figref}[1]{Figure~\ref{#1}}
\newcommand{\newfig}[1]{Fig.~\ref{#1}}
\newcommand{\newfigure}[1]{Figure~\ref{#1}}
\newcommand{\figlegend}[2]{Fig.~\ref{#1}: {#2}}

%% \showfig{file}{label}{[ABC]} 
%% eg \showfig{figs/d94apr26.05.id}{2cell}{} 
\newcommand{\showfig}[2]{
   \pagebreak
   \begin{figure}[htb]
   \includegraphics[scale=1,clip]{#1} 
   {\noindent \bf Fig.~\ref{#2}}
   \end{figure}
}
%% \showbmp{file}{label}{[ABC]} 
%% same as showfig but needed for bitmaps so don't stretch them
\newcommand{\showbmp}[3]{
   \pagebreak
   \begin{figure}[htb]
   \centerline{Figure~\ref{#2}#3}
   %%%% \epsfig{figure=#1}
   \end{figure}
}
%% show the top of 2 or more figures on 1 page
\newcommand{\showfiga}[3]{
   \pagebreak
   \begin{figure}[htb]
   \epsfig{figure=#1,height=#2in,width=#3in}
   \end{figure}
}
%% show the last of 2 or more figs on the page
\newcommand{\showfigb}[4]{
   \begin{figure}[htb]
   \centerline{Figure~\ref{#2}
   \epsfig{figure=#1,height=#3in,width=#4in}}
   \end{figure}
}
%% \showfigc{file}{label}{caption} 
%% eg \showfigc{figs/d94apr26.05.id}{2cell}{The caption} 
\newcommand{\showfigc}[3]{
   \pagebreak
   \begin{figure}[htb]
   \caption{#3}
   \centerline{Figure~\ref{#2}
   \epsfig{figure=#1,height=9in,width=7in}}
   \end{figure}
}

%% \figl{file}{label}{caption} 
%% eg \figl{figs/d94apr26.05.id}{2cell}{The caption} 
\newcommand{\figl}[3]{
   \begin{figure}[htb]
   \begin{center}
   \epsfig{figure=#1,height=3.5in}
   \caption{#3}\label{#2}
   \end{center}
   \end{figure}
}

%% redo for use with graphicx
%% \figi{label}{caption} -- file is in figs/label.id
%% eg \figi{2cell}{The caption} 
\newcommand{\figi}[2]{
  \begin{figure}
    \refstepcounter{figure}
    \label{#1}
    \begin{center}
      \includegraphics[scale=.5,clip]{#1} 
    \end{center}
    {\noindent \bf Fig.~\ref{#1}: #2} 
  \end{figure}
}

%% \figj{label}{caption} -- file is in figs/label.jpg
\newcommand{\figj}[2]{
  \begin{figure}
    \refstepcounter{figure}
    \label{#1}
    \begin{center}
      \includegraphics[scale=.5,clip]{#1.jpg} 
    \end{center}
    {\noindent \bf Fig.~\ref{#1}: #2} 
  \end{figure}
}

%% rotate
%% \figr{label}{caption} -- file is in figs/label.id
%% eg \figr{2cell}{The caption} 
\newcommand{\figr}[2]{
   \refstepcounter{figure}
   \begin{figure}[htb]
   %% \begin{figure}
   \begin{center}
   \label{#1}
   \includegraphics[scale=.5,angle=270,clip]{#1} 
   \end{center}
   % \caption{#2}
   {\noindent \bf Fig.~\ref{#1}: #2}
   \end{figure}
}

%% allow choice of scaling
\newcommand{\figrs}[3]{
%%   \begin{figure}[htb]
   \begin{figure}
   \refstepcounter{figure}
   \begin{center}
   \label{#1}
   \includegraphics[scale=#2,angle=270,clip]{#1} 
   \end{center}
   % \caption{#3}
   {\noindent \bf Fig.~\ref{#1}: #3}
   \end{figure}
}

%% scaling + rotation
\newcommand{\figs}[3]{
%%   \begin{figure}[htb]
   \begin{figure}
   \refstepcounter{figure}
   \label{#1}
   \begin{center}
   \includegraphics[scale=#2,clip]{#1} 
   \end{center}
   % \caption{#3}
   {\noindent \bf Fig.~\ref{#1}: #3}
   \end{figure}
}

% figos -- without the begin{figure} end{figure}
\newcommand{\figos}[3]{
   \begin{center}
   \label{#1}
   \includegraphics[scale=#2,clip]{#1} 
   \end{center}
   {\noindent \bf Fig.~\ref{#1}: #3}
}

% ig -- for making slides
\newcommand{\ig}[2]{
   \begin{center}
   \includegraphics[scale=#2]{#1} 
   \end{center}
}

% igr -- rotate ig
\newcommand{\igr}[2]{
   \begin{center}
     \vspace*{-0.2in}
   \includegraphics[scale=#2,angle=270]{#1} 
   \end{center}
}

% testing
\newcommand{\been}[0]{\begin{enumerate}}
\newcommand{\enen}[0]{\end{enumerate}}
\newcommand{\choices}[5]{\begin{enumerate} \item{#1} \item{#2} \item{#3} \item{#4} \item{#5} \end{enumerate}}

%%% Local Variables: 
%%% mode: latex
%%% TeX-master: t
%%% End: 

In any complex system, only a small amount of data is observable. We are limited by temporal scale -- neural systems are history-dependent (non-Markovian) but we can only
observe for a limited amount of time. We are also limited by spatial scale: 
small local phenomenon such as postsynaptic potentials in spines remain unavailable.

Rate codes provide an envelope for temporal codes. A high rate of coding is evidence that a neuron is active and is working to convey information to other neurons. This rate
can by itself be interpreted as a code at either the single neuron or the ensemble level. However, this interpretation is that of an observer, the experimenter, and does not
tell us how, where or why the code is being used internally.[o][p][q][r][s] While this may be very valuable for designing a brain machine interface, the observer
interpretation is not necessarily useful in figuring out how the brain operates. From the point of view of theoretical neuroscience, a priority should be to decipher how the
biophysical downstream elements extract information from a spike train rather than on how an external observer could do so. Having identified the neuron as carrying
(functional) information, we can then ask how that information is being carried during the period of rapid firing. This is where we can identify a number of time-based codes,
including but not limited to: wave-front codes, synchrony codes, sequence/synfire codes[t], and oscillation phase codes.

Sometimes temporal coding refer to the encoding of temporal aspects of stimulus signal (Theunissen and Miller, 1995), so we need to be careful about the choice of word
(spike-based or time-based?)
Adapted from Gasparini and Magee 2006. Spatially clustered inputs have a lower threshold to evoke an action potential in a CA1 pyramidal neuron. If those clustered inputs are
also synchronous within a narrow temporal window, the threshold is even lower.

%% ideally no disassembly
Let's consider a modern (computerized) internal combustion engine as an example of an engineered system that utilizes explicit signals to manage fuel injection, rotation,
valves, etc. In an auto diagnostic shop, we examine signals and signs from the engine.
Ideally we want to study the system without disassembly since tearing the system apart will destroy the dynamics, signals, and codes that we want to study.
For example, in examining the functioning of an internal combustion engine, one looks at epiphenominal externalities to identify problems. Examination of vibrations will provide
evidence of timing inefficiencies; examination of the exhaust will speak to incomplete burning of fuel; examination of oil will show evidence of piston wear.
Because the engine is a product of our engineering, these epiphenomena can be reliably interpreted as evidence of order or disorder in the underlying system. In the case of an
engine, we can also completely tear it apart and microscopically examine for further signs that would be associated with dysfunction.
